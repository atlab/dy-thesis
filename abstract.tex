Current standard models of the brain's sensory circuits are derived from and supported by detailed empirical characterizations of the responses of individual neurons to external stimuli.
Single-unit statistical descriptions of neurons' stimulus response properties such as, for example, the \emph{receptive field} and \emph{feature selectivity} have proven immensely successful in defining the \emph{functional architecture} of neural circuits and systems: the relationship between response properties of individual cells and the anatomical organization of the circuits  \citep{Ohki:2005,Reid:2012}.
Yet neuroscientists have long recognized that the more intriguing brain functions must be expressed in the dynamic, recurrent interactions between neurons and in the emergent, intrinscially generated population activity that is not reducible to stimulus-driven response \citep{Yuste:2005}.
This recognition motivates ambitious technological developments to record the simultanous spiking activity of large populations of neurons in intact neural circuits \emph{Alivisatos:2013}. 
It is often taken for granted that large-scale recordings of the spiking activity of cells large-scale recordings of the spiking activity of cells 
