When statistical descriptions of populations activity are presented, inevitably, the question arises, ``But what does this mean?'' 
Do the interactions in functional connectivity models translate to anatomical, synaptic connections and if not, then what?  Sometimes an improved statistical model of population activity simply means that we can more reliably read out  the information encoded by the population \citep{Pillow:2011}. However, in this dissertation, we specifically set out to measure statistical expressions of interactions between cells. 
 
Instructive in this regard is the success story of receptive field and feature selectivity of single cells. 
A fortuitous fact of neurobiology is that the spiking activity of individual cells can be efficiently described as a function of external stimuli by simple statistical models with only a handful of parameters \citep{Carandini:2005}. From the original discovery and with subsequent refinement of such descriptions over the past 50+ years, constructs such as \emph{receptive fields} and \emph{orientation tuning} have become principal tools for defining the \emph{functional architecture} of sensory cortex, \emph{i.e.}\ the relationship between single-cell properties related to neural function and the cell types, cortical layers, synaptic connectivity, and the physical arrangements of cells  \citep{Hubel:1962,Ohki:2005,Reid:2012}. Although receptive fields did suggest a mechanistic explanation, the immediate validation of receptive fields as a fundamental principle of neural computation came from finding intricate relationships between the tuning properties of cells and circuit anatomy. Tuning properties of cells were organized into maps, columns, and pinwheels. 

Similarly, a successful metric of functional connectivity need not estimate the exact synaptic connectivity in the circuit from its spiking activity --- quite likely an intractable task --- but extract a simple succinct summary of the activity that relates function to anatomy, constrains theoretical models of underlying computations, and feeds hypotheses of  physiological and computation principles.

 Starting with the intuition that conditoning the activity of pairs of variables on all the other measured variables should improve estimation of their conditional independence or absence of direct interaction, we tested the hypothesis that partial correlations, estimated with a numerically stable method, should yield a closer reflection of physiological interactions than the currently prevalent noise correlations. In the first study described in Chapter 2, we performed empirical cross-validation of several regularization schemes to select the optimal estimator of partial correlations for calcium signals in cortical populations. 

In the second study (Chapter 3), we found new, robust, and nontrivial relationships between the distributions of partial correlations and the anatomical organization of the local circuit. Some of these effects may have intuitive explanations such as knowne  differential  rates of directed inhibitory or excitatory projections.  Others may suggest multiple explanations to be resolved in future experiments. In ongoing experiments, we are estimating functional connectivity in cortical populations with labeled somatostatin-positive neurons and ontogenetically related groups of neurons. Relating studies of functional connectivity to studies of synaptic connectivity and functional architecture of the neocortex will help form a more complete understainding of cortical computations.
