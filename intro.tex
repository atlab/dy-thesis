\section{Activity in cortical microcircuits}
Little is known about the operating principles of the cerebral cortex underlying its immense capabilities. 
Understanding these principles holds promise of treating neurological disorders and even reproducing brain function in artificial neural networks.
The cerebral cortex is hypothesized to comprise smaller modules, replicated across the brain, that implement canonical computations adaptable to solving different problems in each brain region \citep{Mountcastle:1997, Douglas:2007}. 
Thus the path to understanding cortical computations must begin by understanding canonical computations in local microcircuits.

Previous studies of cortical microcircuits have revealed locally dense recurrent connectivity \citep{Douglas:2007} shaping stereotyped dynamic activity patterns spontaneously or in response to stimulation \citep{MacLean:2005,Luczak:2009,Harris:2011,Miller:2014}. These circuit properties suggest the importance of local interactions in defining the activity of individual cells, consistent with the idea of modular design. 

A serious limitation of previous studies has been the sparse sampling of neurons in the circuit.  Two-photon imaging of somatic calcium signals has emerged as the method of choice for studying local microcircuits \citep{Stosiek:2003,Ko:2011,Ko:2013,Hofer:2011}.  Recent advances have made it possible to simultaneously record the activity of nearly every cell in a volume of tissue \emph{in vivo} \citep{Reddy:2005, Katona:2012, Cotton:2013}. 

\section{Functional connectivity in cortical microcircuits}
The term \emph{functional connectivity} has been used to denote different things in the literature. Here we define functional connectivity as a statistical description of the joint activity 

%:w A fortuitous fact of neurobiology is that the spiking activity of individual cells can we efficiently described as a function of external stimuli with simple statistical models with only a handful of parameters fitted to the data \citep{Carandini:2005}. From the original discovery and with subsequent refinement of such descriptions over the past 50+ years, constructs such as \emph{receptive fields} and \emph{orientation tuning} have become principal tools for defining the \emph{functional architecture} of sensory cortex, \emph{i.e.}\ the relationship between single-cell properties of neural function and the cell types, cortical layers, synaptic connectivity, and the physical arrangements of cells  \citep{Hubel:1962,Ohki:2005,Reid:2012}.
