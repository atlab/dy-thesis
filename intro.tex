\section{Single-cell descriptions of neural function}
A fortuitous fact of neurobiology is that the spiking activity of individual cells can we efficiently described as a function of external stimuli with simple statistical models with only a handful of parameters fitted to the data \citep{Carandini:2005}. From the original discovery and with subsequent refinement of such descriptions over the past 50+ years, constructs such as \emph{receptive fields} and \emph{orientation tuning} have become principal tools for defining the \emph{functional architecture} of sensory cortex, \emph{i.e.}\ the relationship between single-cell properties of neural function and the cell types, cortical layers, synaptic connectivity, and the physical arrangements of cells  \citep{Hubel:1962,Ohki:2005,Reid:2012}.

\section{Functional connectivity in multineuronal recordings}
Despite the remarkable success of single-cell descriptions of neural functions, neuroscientists have long recognized that the truly intriguing properties of brain function emerge in the interactions of cells between each other and the resulting intrinsically generated activity \citep{Yuste:2005}.  A major obstacle to investigating functional interactions in populations on neurons, 

\section{Functional connectivity in population calcium signals}
