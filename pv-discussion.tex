\section{Discussion}

Although this chapter presents preliminary results with only 11 analyzed datasets, several intriguing findings begin to emerge.
First, we confirmed that regularized partial correlations had stronger correspondence to circuit anatomy than conventional noise correlation when compared by cell type (Fig.~\ref{fig:pv2}), physical distance between the soma (Fig.~\ref{fig:pv3}), and differences in preferred orientations (Fig.~\ref{fig:pv4}). Greater correspondence to anatomy constitutes indirect evidence that partial correlations in densely recorded microcircuit provide a closer approximation of specific direct interactions while correlations reflect the final confounded result of these interaction.

The finding that PV+ cells form dense networks of positive interactions that are not explained by common fluctuations of the entire circuit (Fig.~\ref{fig:pv2} D) runs against a prevailing view that the PV+ subnetwork is highly synchronized, acting as a single unit. Several physiological mechanisms can be proposed to explain such strong sparse  pairwise partial correlations. For example, the PV+ subnetwork could be receiving  elaborate (not uniform or strongly divergent) input from layers 5.  These interaction could also be mediated by gap junctions previously shown to exist between PV+ cells \cite{Tamas:2000}. 

The finding that PV+/PV+ and PV--/PV-- pairs both form stronger interactions than PV--/PV+ pairs (Fig.~\ref{fig:pv2} B, D, and F) suggests that PV+ neurons form a relatively isolated subnetwork. 

The finding of robustly negative average partial correlations for PV--/+ pairs separated by more than 100 $\mu$m (Fig.~\ref{fig:pv3} B), if further confirmed, would constitute, to our knowledge, the first description of a consistently negative correlation between anatomically identified cell groups. 
The cortical distance at which such negative correlations emerge could serve to delineate the spatial extent of functional modules in the cortex.

The finding that PV--/PV-- partial correlations are selective to $\Delta$ori while PV--/PV+ are not (Fig.~\ref{fig:pv4} B)  is consistent with the pattern of excitatory synaptic connectivity measured by \cite{Hofer:2011}. 
