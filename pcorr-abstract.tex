Ambitious projects aim to record the activity of ever larger and denser neuronal populations \emph{in vivo}.  Correlations in neural activity measured in such recordings can reveal important aspects of  neural circuit organization.  However, estimating and interpreting large correlation matrices is statistically challenging.  Estimation can be improved by regularization, \emph{i.e.}\;by imposing a structure on the estimate.  The amount of improvement depends on how closely the assumed structure represents dependencies in the data. Therefore, the selection of the most efficient correlation matrix estimator for a given neural circuit must be determined empirically.  Importantly, the identity and structure of the most efficient estimator informs about the types of dominant dependencies governing the system.
We sought statistically efficient estimators of neural correlation matrices in recordings from large, dense groups of cortical neurons.  Using fast 3D random-access laser scanning microscopy of calcium signals, we recorded the activity of nearly every neuron in volumes 200 $\mu$m wide and 100 $\mu$m deep (150--350 cells) in mouse visual cortex.  We hypothesized that in these densely sampled recordings, the correlation matrix should be best modeled as the combination of a sparse graph of pairwise partial correlations representing interactions between the observed neuronal pairs and a low-rank component representing common fluctuations and external inputs.  Indeed, in cross-validation tests, the covariance matrix estimator with this structure consistently outperformed other regularized estimators. The sparse component of the estimate defined a graph of interactions. These interactions reflected the physical distances and orientation tuning properties of cells: The density of positive `excitatory' interactions decreased rapidly with geometric distances and with differences in orientation preference whereas negative `inhibitory' interactions were less selective.  Because of its superior performance, this `sparse + latent' estimator likely provides a more physiologically relevant representation of the functional connectivity in densely sampled recordings than the sample correlation matrix.