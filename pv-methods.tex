\section{Methods}
\subsection{Animals}

My fellow member of Andreas Tolias' lab, Cathryn Cadwell, conducted the breeding and genotyping of mice with fluorescent labelling of PV+ interneurons.

The PV-Cre and Ai9 mouse lines were obtained from the Jackson Laboratory (Stock \#017320 and \#007909, respectively) and maintained as separate homozygous lines in our lab.  To generate double heterozygous animals carrying one copy of each allele, homozygous PV-Cre males were crossed with homozygous Ai9 females. Two males and two females aged 66, 67, 68, and 235 days were used in these experiments. 

Some animals were genotyped to confirm that they carried the appropriate alleles. Tail samples for genotyping were taken under isoflurane anesthesia on or before p28.  DNA extraction was performed in 25 mM NaOH, 0.2 mM EDTA solution at 95$^\circ$C for 30 minutes followed by neutralization with an equal volume of 40 mM Tris HCl.  PCR was carried out using primers for Cre (Cre F: 5'-GCATTA\-CCG\-GTC\-GAT\-GCA\-ACG\-AGT\-GATGAG-3'; Cre R: 5'-GAGTGA\-ACG\-AAC\-CTG\-GTC\-GAA\-ATC\-AGTGCG-3') or primers for Ai9 (Ai9 F: 5'-GTAATG\-CAG\-AAG\-AAG\-ACT\-ATG\-GGC\-TGGGAG-3'; Ai9 R: 5'-ATGTCC\-AGC\-TTG\-GAG\-TCC\-ACG\-TAG\-TAGTAG-3') using a thermal cycler (Eppendorf Mastercycler pro S) with the following program: 94$^\circ$C for 50 s; 30 cycles of 94$^\circ$C for 20 s, 60$^\circ$C for 45 s, and 68$^\circ$C for 2 min; 68$^\circ$C for 1 min; and 10$^\circ$C $\infty$.

\subsection{Data acquisition and processing}
The visual stimulus protocol and  data acquisition procedures  were identical to those described in Chapter 2. PV+ cells were visually identified on the red channel and marked during the semiautomatic cell segmentation process.
