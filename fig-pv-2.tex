\begin{figure}
\begin{leftfullpage}
\caption[Functional connectivities for PV--/--, PV--/+, and PV+/+ cell pairs]
{{\bf Functional connectivities for PV--/--, PV--/+, and PV+/+ cell pairs}

Top panels (A,C, and E) show aspects of functional connectivity expressed through conventional noise correlations. 
Bottom panels (B, D, and F) show connectivity expressed through regularized partial noise correlations.
Data points represent averages for each of $n=11$ sites conditioned on cell pair type.
Overall, partial noise correlations provide stronger effects and greater discriminability of cell pair types.

{\bf A.} Average noise correlations. 

{\bf B.} Average partial noise correlations with regularization.  
The relationship to the cell pair types is clearer and more consistent for average partial correlations than for noise correlations.

{\bf C.} Rates of positive (green) and (negative) connectivity obtained by thresholding the correlations to make a sparse matrix of interactions.  
The sparsities was matched to matrices in panel D.

{\bf D.} Rates of positive (green) and (negative) connectivity in the sparse component of the partial correlation estimates.

{\bf E.} Fraction of negative interactions when interactions are obtained by thresholding noise correlations as in panel C. 

{\bf F.} Fraction of negative interactions when interactions are obtained by from the sparse component of the sparse+latent covariance estimator as in panel D.
The relationship of connectivity rates to the cell pair types is clearer and more consistent in the sparse component of the sparse+latent estimator than for thresholded correlations.


}\label{fig:pv2}
\end{leftfullpage}
\end{figure}

\begin{figure}
\begin{fullpage}
\begin{center}
    \includegraphics[width=\textwidth]{./figures/pv-fig2.pdf}
\end{center}
\end{fullpage}
\end{figure}
