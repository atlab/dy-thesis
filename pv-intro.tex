\section{Introduction}
Inhibitory parvalbumin-expressing (PV+) interneurons represent a relatively small fraction of cortical neurons but play critical roles in coordinating the activity of cortical microcircuits. 
By targeting the perisomatic or axonal domains of nearly all principal cells in their near neighborhood within the same cortical layer and receiving reciprocal input from the same population, PV+ cells are uniquely positioned to exert strong shunting inhibition on all local cells, synchronizing their activity and orchestrating oscillations. 
The subcellular distribution of their K$^+$ and Na$^+$ active conductances makes their design optimal for high speed, temporal precision, and selectivity to spatially distributed rather than concentrated input. 
These properties suggest their role as network stabilizers. 
They have also been shown to mediate local network computations such as feedback and forward inhibition, divisive input normalization, and ``winner-take-all'' computations. 
For a thorough review of the design and function of PV+ interneurons, refer to \cite{Hu:2014}.

The functional connectivity of PV+ neurons within cortical microcircuits has only recently been studied by \cite{Hofer:2011}. Using two-photon imaging of calcium signals and genetically encoded fluorescent markers of PV+cells, they found stronger correlations between the activity of PV+ cells than between between PV-- cells in local populations of neurons. The patterns of correlations, \ie their relative magnitudes, were also more preserved across stimulus condition between PV+ cells than between PV-- cells and between PV+/PV-- pairs. These results are consistent with a distinct role of PV+ neurons in stabilizing and orchestrating the population response of the local circuit.

We hypothesized that improved measures of functional connectivity such as the partial correlation estimation developed in \cite{Yatsenko:2014} and described in Chapter 2 would provide greater differentiation of the functional connectivity of PV+ cells than the noise correlations obtained by \cite{Hofer:2011} and that these measurements would suggest specific physiological interactions of the PV+ interneuron network. 
To test this hypothesis, we used the stimulus design and signal acquisition protocol from \cite{Yatsenko:2014} in mice with PV+ neurons labeled with the genetically encoded red fluorescent protein tdTomato. 
We found significantly greater differentiation of the functional connectivity of PV+ cells using regularized partial noise correlations compared to conventional noise correlations.  Surprisingly, PV+ neurons exhibited strong positive partial correlations between themselves consistent with either strong and targeted direct interactions or specific patterns of shared input from outside the recorded circuit. Partial correlations between PV-- and PV+ cells decreased quickly with physical distance and, for distances greater than 100 $\mu$m, became weakly negative, consistent with lateral inhibition of neighboring circuits. Partial correlations had stronger and clearer dependencies than noise correlations on differences in orientation tuning. This was true for both PV--/PV+ and PV--/PV-- cell pairs without clear differences between them. Overall, regularized partial correlations produced a clearer picture of functional connectivity than conventional noise correlations, suggesting hypotheses for their physiological origin to be tested in future experiments.
